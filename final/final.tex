\documentclass{article}
\usepackage[ruled, linesnumbered]{algorithm2e}
\def\showtopic{Numerical Analysis}
\def\showtitle{Solutions to Final Exam}
\def\showabs{Final}
\def\showauthor{Ting Lin, 1700010644}
\def\showchead{LIN}
\usepackage{amsmath, amsfonts, amsthm}

\usepackage{graphicx, epstopdf}
\usepackage{color}
\usepackage{geometry, graphicx}
\usepackage{algorithm, algorithmic}
\usepackage{bm}
\usepackage{multirow}
\usepackage{ulem}
\geometry{left = 5em, right = 5em}
\usepackage{listings}
\usepackage{xcolor}
%% notation macro
\newcommand{\F}{\mathcal F}
\newcommand{\T}{\mathcal T}
\newcommand{\I}{\mathcal I}
\newcommand{\U}{\mathcal U}
\newcommand{\R}{\mathbb R}
\renewcommand{\P}{\mathcal P}
\newcommand{\uP}{ \mathcal \uline P}
\newcommand{\B}{\mathcal B}
%\newcommand{\R}{\mathbb R^2}
\newcommand{\Z}{\mathbb Z}
\newcommand{\C}{\mathbb C}
\newcommand{\laplacian}{\triangle}
\newcommand{\grad}{\nabla}
\renewcommand{\div}{\textrm{div~}}

\newcommand{\diff}[2]{\frac{\partial #1}{\partial #2}}
\newcommand{\difff}[3]{\frac{\parial #1^2}{\partial #2 \partial #3}}
\newcommand{\diFF}[2]{\frac{\partial #1^2}{\partial^2 #2}}
\newcommand{\diam}{\text{ diam }}
%% non-noation macro
\newcommand{\IN}{\text{  in  }}
\newcommand{\ON}{\text{  on  }}
\newcommand{\st}{\text{s.t.  }}
\newcommand{\tbc}{{\color{red}[TBC]}}

%% enviorment
\newtheorem{proposition}{Proposition}
\newtheorem{definition}{Definition}
\newtheorem{corollary}{Corollary}
\newtheorem{remark}{Remark}



\title{\textbf{\showtitle}}
\author{\showauthor}
\usepackage{indentfirst}
\usepackage{fancyhdr}  
\pagestyle{fancy}
\lhead{\textbf {\showtopic} }
\chead{} 
\rhead{\textbf {\showabs} }
\lfoot{} 
\cfoot{\thepage}
\rfoot{} 
\renewcommand{\headrulewidth}{0.4pt} 
\DeclareMathOperator{\size}{size}
\begin{document}
	\maketitle
	\thispagestyle{fancy}
	\tableofcontents
	
	\section*{}

\section{~}
\subsection{~}
\paragraph{Lagrange's Formula:}
\begin{equation}
\begin{split}
P(x) = & 0\frac{(x+0.5)(x-0.5)}{(-1+0.5)(-1-0.5)}  -
1\frac{(x+1)(x-0.5)}{(-0.5+1)(-0.5-0.5)} + 
1\frac{(x+1)(x+0.5)}{(0.5+1)(0.5+0.5)}\\
= & \frac{8}{3} x^2 + 2x - \frac{2}{3}
\end{split}
\end{equation}
\paragraph{Newton's Formula:}
We first calculate that 
$f[-1;-0.5] = -2, f[-0.5;0.5] = 2$
$f[-1;-0.5;0.5] = \frac{8}{3}.$
The by Newton's formula we calculate the interpolation polynomial:
\begin{equation}
\begin{split}
P(x) = & f[-1] + f[-1;-0.5](x+1) + f[-1;-0.5;0.5](x+1)(x+0.5) \\ 
= & 0 - 2(x+1)  + \frac{8}{3}(x+1)(x+0.5) \\
= & (\frac{8}{3}x - \frac{2}{3})(x+1)
\end{split}
\end{equation}

\subsection{~}
The piecewise interpolation function 
\begin{equation}
P(x) = \left\{ 
\begin{aligned}
-2(x+1) & ~~-1\le x \le -0.5 \\
2x &~~ -0.5 \le x \le 0.5
\end{aligned}
 \right.
\end{equation}
And interpolation value become vacuous when $x$ is outside the interval $[-1,0.5]$.


\subsection{~}

The basis of spline function is 
$$1, x+0.5, (x+0.5)^2, (x+0.5)^3, (x+0.5)_+^3.$$
Here $a_+ =\max(a,0)$.
We suppose the spline function has the form 
$$P(x) = a_1(x+0.5)_+^3 + a_2(x+0.5)^3 + a_3(x+0.5)^2 + a_4(x+0.5) + a_5.$$
Then five conditions 
$$P(-1) = 0, P(-0.5) = -1, P(0.5)=1, P''(-1) = 0, P''(0.5) = 0$$ become
\begin{align}
-\frac 18 a_2 + \frac 14 a_3 - \frac 12 a_4 + a_5 =&~ 0\\
a_5 =&~-1\\
a_1 + a_2 + a_3 + a_4 + a_5 =&~1\\
-3a_2 + 2a_3 =&~0\\
6a_1 + 6a_2 + 2a_3 =&~0
\end{align}
Solve the system we find 
$$a_1 = -4, a_2 = \frac 83, a_3 = 4, a_4 = -\frac 23, a_5 = -1,$$
and therefore the spline function is 
$$P(x) = -4(x+0.5)_+^3 + \frac 83(x+0.5)^3 + 4(x+0.5)^2 + -\frac 23(x+0.5) -1 .$$
Again, the function value is vacuous when $x$ outside $[-1,0.5]$, as we indicated in previous subproblem.
\subsection{~}

The following least square problem is
$$\Phi(a,b) = (-a+b)^2 + (-0.5a+b+1)^2 + (0.5a+b-1)^2 = (3a^2)/2 - 2ab - 2a + 3b^2 + 2.$$
We aim to find $a,b$ minimizing $\Phi(a,b)$. 
The first order condition yields
$$3a-2b-2=0, 6b-2a=0$$
The solution is $a = \frac 67, b = \frac 27$. The linear function is $\frac 67 x + \frac 27$.

\section{~}
\subsection{~}
Taking $f(x) = 1, x, x^2$ we have 
\begin{equation}
\label{1}
A_1 + 2 + A_3 = 3 
\end{equation}
\begin{equation}
\label{x}
2x_1 + 3A_3 = \frac 92
\end{equation}
\begin{equation}
\label{x2}
2x_1^2 + 9A_3 = 9
\end{equation}




We have (\eqref{x2} - 3*\eqref{x}):

\begin{equation}
\label{x2-3x}
2(x_1^2-3x_1) = -\frac 92
\end{equation}
yielding that $x_1 = \frac 32$.
Then we know $A_3 = \frac 12$ and $A_1 = \frac 12$. And the scheme is just Simpson scheme. For $f(x) = x^3$, we have $$LHS = \frac{81}{4} = 2(\frac 32)^3 + \frac 123^3 = RHS$$
But for $f(x) = x(3-x)(x-\frac 32)^2$, LHS is non-zero but RHS is zero. Hence the algebraic order of the given scheme is 3.

\subsection{~}
Taking $f(x) = 1, x, x^2, x^3$ we have 
\begin{equation}
\label{11}
A_1 + A_2 + A_3 +A_4= 3 
\end{equation}
\begin{equation}
\label{xx}
\frac{1}{2}A_2 + 2A_3 + 3A_4 = \frac 92
\end{equation}
\begin{equation}
\label{xx2}
\frac{1}{4}A_2 + 4A_3 + 9A_4 = 9
\end{equation}
\begin{equation}
\label{xx3}
\frac{1}{8}A_2 + 8A_3 + 27A_4 = \frac{81}{4}
\end{equation}

This linear system \eqref{11}, \eqref{xx}, \eqref{xx2}, \eqref{xx3} has a unique solution 
$$A_1 = 0, A_2 = \frac{6}{5}, A_3 = \frac{3}{2}, A_4 = \frac{3}{10}$$
And it is self-evident that the scheme is of third order from our deduction.


\section{~}
Rewrite the equation as $$\frac 12(x + \frac ax) = x.$$ We aim to find the fixed point of 
$$\frac{1}{2}(x+ \frac ax).$$


We will show that 
\begin{equation}
\label{iter}
x_{k+1} = \frac 12 (x_k + \frac{a}{x_k})
\end{equation} is locally convergent. Direct calculation yields
\begin{equation}
\label{iter2}
\begin{split}
x_{k+1} - \sqrt{a}
 ~=~& \frac 12 (x_k - \sqrt{a} + \frac{a}{x_k} - \sqrt{a})\\
~=~& (x_k - \sqrt a)(\frac 12 - \frac 12 \frac{\sqrt a}{x_k})\\
~=~& \frac{1}{2x_k} (x_k - \sqrt a)^2
\end{split}
\end{equation}
Then if $|x_k - \sqrt{a}|<\frac{1}{4} \sqrt a$, we obtain that $$|x_{k+1} - \sqrt{a}| \le  \frac 16 |x_k - \sqrt a|$$
Hence the scheme converges locally. 

From \eqref{iter2}, we conclude that the scheme is of second order, and the error constant $$\lim_{n\to \infty} \frac{|x_{k+1} - \sqrt a|}{|x_{k} - \sqrt a|^2} = \frac{1}{2\sqrt a}$$
\section{~}
\subsection{~}
We expand LHS and RHS resp. and for simplicity we use $f, f[x], f[y]$ to denote $f(x_n, y_n), f_x(x_n,y_n), f_y(x_n,y_n)$.
\paragraph{LHS}
\begin{equation}
\begin{split}
y_{n+1} =& y_n + hf + \\
&+\frac{1}{2}h^2(f[x] + f[y]y') \\
&+ \frac{1}{6}h^3(f[xx] + 2f[xy]y' + f[yy]y'^2 + f[y]y'') + O(h^4)
\end{split}
\end{equation}
Noticing that $y' = f(x,y(x)) = f, y'' = f[x] + f[y]y' = f[x] + ff[y]$, we have 
\begin{equation}
\begin{split}
y_{n+1} =& y_n + hf + \\
&+\frac{1}{2}h^2(f[x] + f[y]f) \\
&+ \frac{1}{6}h^3(f[xx] + 2ff[xy] + f^2f[yy] + f[x]f[y]+ff[y]^2) + O(h^4)
\end{split}
\end{equation}
\paragraph{RHS}
\begin{equation}
\begin{split}
K_2 =& f + \\
&+ h(ff[y]b_{21} + f[x]a_2) + \\
&+ \frac{1}{2}h^2(f^2f[yy]b_{21}^2 + 2ff[xy]a_2 b_{21} + f[xx]a_2^2) + O(h^3)
\end{split}
\end{equation}
\begin{equation}
\begin{split}
K_3 =& f + \\
&+ h(ff[y]b_{31} + ff[y]b_{32} +  f[x]a_3) + \\
&+ \frac{1}{2}h^2(f^2f[yy]b_{31}^2 + 2f^2f[yy]b_{31}b_{32} + f^2f[yy]b_{32}^2 +\\
&+2ff[y]^2 b_{21}b_{32} + 2ff[xy]a_3b_{31} + 2ff[xy]a_3b_{32} + 2f[x]f[y] a_2b_{32}+ f[xx]a_3^2) + O(h^3)
\end{split}
\end{equation}
Hence 
\begin{enumerate}
	\item Comparing $h$: $c_1 + c_2 + c_3 = 1$
	\item Comparing $h^2$ term:
		$f[x]$ term: $$c_2a_2 + c_3a_3 = \frac{1}{2}$$
		$ff[y]$ term: $$c_2b_{21} + c_3b_{31} + c_{3}b_{32} = \frac{1}{2}$$
	\item Comparing $h^3$ term:
		$f[xx]$ term:
			$$c_2a_2^2 + c_3a_3^2 = \frac{1}{3}$$
		$ff[xy]$ term:
		$$2c_2a_2b_{21} + 2c_3a_3b_{31} + 2c_3a_3b_{32} = \frac{2}{3}$$
		$f^2f[yy]$ term:
		$$c_2b_{21}^2 + c_3b_{31}^2 + 2c_3b_{31}b_{32} + c_3 b_{32}^2 = \frac 13$$
		$f[x]f[y]$ term:
		$$2c_3a_2b_{32} = \frac 13$$
		$ff[y]^2$ term:
		$$2c_3b_{21}b_{32} = \frac 13$$
\end{enumerate}

The eight equations are what we need.
\subsection{~}
We choose 
$$a_2 = \frac 12 , a_3 = \frac 34$$
$$c_{1} = \frac 29, c_2 = \frac 13, c_3 = \frac 49$$
$$b_{21} = \frac 12, b_{31} = 0, b_{32} = \frac 34$$

And it is a solution to equations in 4.1.

\subsection{~}
From 4.1 and 4.2, we have already known that $y(x_{n+1}) - y(x_n) = O(h^4)$. Based on this, we prove the scheme is of third order. 
We suppose our scheme is written as 
$$y_{n+1} = y_n + h\phi(x_n,y_n,h)$$ We first prove the result provided $\phi$ is Lipschitz on $y$. 
\begin{proof}
	We define $\bar {y}_{n+1} = y(x_n) + h\phi(x_n, y(x_n),h)$,then we have $|\bar{y}_{n+1} - y(x_{n+1})| \le Ch^4$ for some constant $C$ (independent on $h$). 
	
	Set $e_n = |y_n - y(x_n)|$, we found that 
	\begin{equation}
	\begin{split}
	e_{n+1} =& |y_{n+1} - y(x_{n+1})|\\
	\le &|y_{n+1} - \bar y_{n+1}| + |\bar y_{n+1} - y(x_n)| \\
	\le & |y_n + h\phi(x_n,y_n,h) - y(x_n) - h\phi(x_n, y_n, h)| + Ch^{4}\\
	\le & (1+hL)e_n + Ch^4.
	\end{split}
	\end{equation}
	Here $L$ is the Lipschitz constant of $\phi$.
	
	
	We have 
		\begin{equation}
	\begin{split}
	e_{n+1} \le& (1+hL)e_n + Ch^4 \\
	\le& (1+hL)e_{n-1} + C[1 + (1+hL)]h^4\\
	\le & \cdots \\
	\le & (1+hL)^n e_0 + C\frac{(1+hL)^n-1}{hL}
h^4\\
\le & \frac{C}{L}h^3(e^{nhL}-1)	\end{split}
	\end{equation}
	Since $nh$ is the length of time interval and hence a fixed number, we conclude the result.
\end{proof}
It suffices to show that $\phi$ is Lipschitz on $y$, and the constant is not dependent on $h<1$. In fact, we show this is correct for all Runge--Kutta formula, provided $f$ is Lipschitz on $y$.

Suppose that $|f(x,y) - f(x,y')| \le L_f|y-y'|$, we have
\begin{equation}
|K_1(x_n,y_n) - K_1(x_n,y_n')| \le L_f|y-y'|
\end{equation}

\begin{equation}
%\begin{split}
|K_2(x_n,y_n) - K_2(x_n,y_n')| \le L_f|y-y'| + L_f|hb_{21}||K_1(x_n,y_n) - K_1(x_n,y_n')| 
\le  [L_f + L_f^2|b_{21}]|y-y'|
%\end{split}
\end{equation}
\begin{equation}
\begin{split}
|K_3(x_n,y_n) - K_3(x_n,y_n')| \le& L_f|y-y'| + L_f|b_{31}||K_1(x_n,y_n) - K_1(x_n,y_n')| + L_f|b_{32}||K_2(x_n,y_n) - K_2(x_n,y_n')|\\
\le & \Big[L_f + L_f^2|b_{31}| + L_f^2|b_{32}| + L_f^3|b_{32}b_{21}|\Big]|y-y'|
\end{split}
\end{equation}

Hence we conclude that $\phi(x,y,h)$ is Lipschitz on $y$, with constant $3L_f +  L_f^2|b_{21}+L_f^2|b_{31}| + L_f^2|b_{32}| + L_f^3|b_{32}b_{21}|$.
 
\end{document}
















Escape special TeX symbols (%, &, _, #, $)